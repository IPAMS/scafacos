\chapter{P3M -- Particle-Particle Particle-Mesh Ewald}
\label{cha:p3m}

\solvertoindex{P3M}
\solvertoindex{P$^3$M}
\solvertoindex{Particle-Particle Particle-Mesh Ewald}
\index{Particle Mesh Ewald}

\section{Features}

\begin{description}
\item[Periodicity:] Only fully periodic boundaries are supported.
\item[Box shape:] Triclinic boxes are supported by the analytical differentiation scheme. Any orthorhombic box shape is supported by the analytical and the ik differentiation scheme. 
\item[Autotuning:] The parameters can be automatically tuned when a
  required tolerance in the absolute rms field error is provided.
\item[Delegate near-field:] Yes.
\item[Virial:] Not yet supported, but in progress.
\end{description}

\section{Solver-specific Parameters}

\begin{itemize}
\item \verb!tolerance_field! The allowed tolerance of the absolute rms
  error in the fields. If the required tolerance can't be achieved
  with the given parameters, the tuning is aborted. Feasible values
  depend on the system that is computed. In an MD simulation with a
  thermostat, this should be about an order of magnitude less than the
  forces generated by the thermostat.
\item \verb!r_cut! Cutoff distance of the near field.  Can be
  automatically tuned.  Feasible values are in the order of the mean
  free path between charges. \verb!r_cut! has to be less than half the
  smallest simulation box length.  When \verb!r_cut! is chosen too
  small, the errors of the algorithm become large, the required
  accuracy might not be achieved, or the grid has to be chosen very
  large, which will slow down the algorithm.  When \verb!r_cut! is
  chosen too large, the algorithm might become slow as most
  computation is done in the badly scaling near field region.
\item \verb!grid! Size of the grid. Can be automatically
  tuned. Feasible values are in the order of $N^{\frac{1}{3}}$, \ie a
  grid point for each particle. The minimal grid size is $4$, the
  maximal grid size is $512$. The larger the grid, the smaller the
  error, but also the higher the memory and computational requirements
  of the algorithm.
\item \verb!cao! ``Charge assignment order'': The number of points in
  each direction that the charge gets smeared out to. Can be
  automatically tuned.  Allowed values are between $1$ and $7$.  The
  larger \verb!cao!, the smaller the error, but also the higher the
  computational cost of the algorithm.
\item \verb!alpha! Ewald splitting parameter. Should be automatically
  tuned. Set this manually only when you know what you are doing.
\end{itemize}

%%% Local Variables: 
%%% mode: latex
%%% TeX-master: ug.tex
%%% End: 

