\chapter{Ewald}
\label{cha:ewald}

\solvertoindex{Ewald}

%%% Local Variables: 
%%% mode: latex
%%% TeX-master: ug.tex
%%% End: 

The well-known Ewald formula for the computation of \eqref{eq:ewald_potential} splits the
electrostatic potential $\phi$ into the following parts
\begin{equation}\label{eq:ewald}
 \phi = \phi^{\text{real}} + \phi^{\text{reci}}  + \phi^{\text{self}}  + \phi^{\text{dipo}},
\end{equation}
where the contribution from real space $ \phi^{\text{real}}$, reciprocal space $\phi^{\text{reci}}$,
the self-energy $\phi^{\text{self}}$ and the dipole correction $\phi^{\text{dipo}}$ are given by
\begin{eqnarray}
  \phi^{\text{real}}(\mathbf x_j)
  &=& \nonumber
    \sum_{\mathbf r\in \mathbb Z^3}
    \underset{ l\neq j \,{\text{for}}\, \mathbf r= \mathbf 0}{\sum_{l=1}^M}
    q_l\frac{{\text{erfc}}(\alpha \|\mathbf x_j -\mathbf x_l +\mathbf r B\|_2)}{\|\mathbf x_j -\mathbf x_l +\mathbf r B\|_2},\\
  \phi^{\text{reci}}(\mathbf x_j)
  &=& \label{eq:ewald_reci}
    \frac{1}{\pi B} \sum_{\mathbf k\in \mathbb Z^3\setminus\{\mathbf 0\}}
    \frac{{\text{e}}^{-\pi^2 \|\mathbf k\|_2^2/(\alpha B)^2}}{\|\mathbf k\|_2^2}
    \sum_{l=1}^M q_l {\text{e}}^{-2\pi \text{i} \mathbf k (\mathbf x_j -\mathbf x_l)/B}\, ,\\
  \phi^{\text{self}} (\mathbf x_j)
  &=&
    -2 q_j \frac{\alpha}{\sqrt{\pi}}\; ,\nonumber  \\
  \widetilde\phi^{\text{dipo}}
  &=&
    \frac{2\pi}{3V}\bigg(\sum_{l=1}^Mq_{l}\mathbf x_{l}\bigg)^{2}\; .\nonumber %\label{eq:E_dipol_part}
\end{eqnarray}
Thereby, the complementary error function is defined by $\textrm{erfc}(z) =
\frac{2}{\sqrt{\pi}}\int_z^\infty \text{e}^{-t^2} \text{d}t$.
Choosing optimal parameters, Ewald summation scales as ${\cal O}(M^{3/2})$ \cite{kolafa92a}.
