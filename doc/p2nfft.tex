\chapter{\ptwonfftcap -- Particle-Particle NFFT}
\label{cha:p2nfft}

\solvertoindex{\ptwonfft}
\solvertoindex{Particle-Particle NFFT}

\def\alp{13}


%%% Local Variables: 
%%% mode: latex
%%% TeX-master: ug.tex
%%% End: 


\ptwonfft is a common framework for almost all FFT based fast Coulomb solvers.
The computation of Coulomb interactions is split into a short range interaction (near field) and
a long range interaction (far field).
For sake of simplicity, we describe the idea of \ptwonfft only for the one-dimensional case.
In addition, we do not stress all the details, that arise from the difference between periodic and
non-periodic boundary conditions. More detailed descriptions of the algorithms can be found in \cite{PiPo10}.

\section{Description of the Method}

As usual, we assume $M$ charges $q_l$ at position $r_l$. The potentials $\phi_j$ at position $r_j$
are given by
\begin{equation*}
  \phi_j
  =
  \sideset{}{^\prime}\sum_{l=1}^M
    q_l\frac{1}{r_{jl}}
  \,,
\end{equation*}
and the Electrostatic fields $E_j$ at position $r_j$ are given by
\begin{equation*}
  E_j
  =
  \sideset{}{^\prime}\sum_{l=1}^M
    q_l\nabla\frac{1}{r_{jl}}
  =
  \sideset{}{^\prime}\sum_{l=1}^M
    q_l \frac{r_l}{r_{jl}}
  \,.
\end{equation*}
Hereby, the distance between two particles at positions $r_j$ and $r_l$ is given by $r_{jl}:=|r_j-r_l|$.

\subsection{Periodic boundary conditions}
In this section we present a straightforward method, that accelerates the traditional Ewald
summation technique by NFFT. This combination was first presented in \cite{HeLa06} and
is very similar to the FFT-accelerated Ewald sums, namely, the so-called
particle-particle particle-mesh (P$^3$M), particle-mesh Ewald (PME) and smooth particle-mesh
Ewald (SPME), see also \cite{DeHo98a}.
Additionally we will see, that the accelerated Ewald summation can be reinterpreted
into a method very similar to our fastsum Algorithm~\ref{algo:fastsum}.


In order to overcome this increase in time we apply the NFFT for the calculation of the
reciprocal-space potential  $\phi^{\text{reci}}$ and we obtain a method similar as our fast summation method.
To this end, we compute the Fourier transformed charge density
\begin{equation*}
 S(\mathbf k) = \sum_{l=1}^M q_l \text{e}^{+2\pi \text{i} \mathbf k \mathbf x_l/B}
\end{equation*}
by NFFT$^\adj$ and after truncation of the sum \eqref{eq:ewald_reci} we obtain by NFFT
\begin{equation*}
 \phi^{\text{reci}}(\mathbf x_j) \approx
 \frac{1}{\pi B} \sum_{\mathbf k\in I_{\mathbf N}\setminus\{\mathbf 0\}}
 \frac{\text{e}^{-\pi^2 \|\mathbf k\|_2^2/(\alpha B)^2}}{\|\mathbf k\|_2^2}
 S(\mathbf k)\text{e}^{-2\pi \text{i} \mathbf k \mathbf x_j/B}\, .
\end{equation*}





\subsection{Calculation of the Potentials}

\begin{figure}[htb]
  \centering
  \beginmypgfhack{p2nfft_ewald_split}
%   \beginpgfgraphicnamed{p2nfft_ewald_split}
  \begin{tikzpicture}
    \node (plot_one_over_r) [right]{
      \begin{tikzpicture}
        \begin{axis}[tiny, xmin=0, xmax=0.5, ymin=0, ymax=17]
          \addplot gnuplot [id=erf,mark=none,domain=0:0.5, samples = 101, very thick, black]{(x==0) ? 100 : 1/x};
        \end{axis}
      \end{tikzpicture}
    };
    \node (plot_equals) at (plot_one_over_r.east) [right]{=};
    \node (plot_erfc) at (plot_equals.east) [right]{
      \begin{tikzpicture}
        \begin{axis}[tiny, xmin=0, xmax=0.5, ymin=0, ymax=17]
          \addplot gnuplot [id=erf,mark=none,domain=0:0.5, samples = 101, very thick, black]{(x==0) ? 100 : 1/x};
          \addplot gnuplot [id=erf,mark=none,domain=0:0.5, samples = 101, very thick, blue]{(x==0) ? 100 : erfc(x*\alp)/x};
        \end{axis}
%         \node [xshift=8ex, yshift=9ex] {\tiny$\alpha=\alp$};
      \end{tikzpicture}
    };
    \node (plot_plus) at (plot_erfc.east) [right]{+};
    \node (plot_erf) at (plot_plus.east) [right]{
      \begin{tikzpicture}
        \begin{axis}[tiny, xmin=0, xmax=0.5, ymin=0, ymax=17]
          \addplot gnuplot [id=erf,mark=none,domain=0:0.5, samples = 101, very thick, black]{(x==0) ? 100 : 1/x};
          \addplot gnuplot [id=erf,mark=none,domain=0:0.5, samples = 101, very thick, red]{(x==0) ? 2*\alp/sqrt(pi) : erf(x*\alp)/x};
        \end{axis}
%         \node [xshift=8ex, yshift=9ex] {\tiny$\alpha=\alp$};
      \end{tikzpicture}
    };
    \node (eqn_one_over_r) at (plot_one_over_r.north) [above]{
      $\tfrac{1}{r}$
    };
    \node (eqn_erfc) at (plot_erfc.north) [above]{
      \color{blue}$\tfrac{1}{r} - R(r)$
    };
    \node (eqn_erf) at (plot_erf.north) [above]{
      \color{red}$R(r)$
    };
    \node (eqn_equals) at (plot_equals.north) [above=7ex]{=};
    \node (eqn_plus) at (plot_plus.north) [above=7ex]{+};
  \end{tikzpicture}
  \endmypgfhack
%   \endpgfgraphicnamed
  \caption{Splitting of Coulomb potential into near field (blue) and far field (red).}
\end{figure}

\begin{equation*}
  \phi(r_j)
  =
  \sideset{}{^\prime}\sum_{l=1}^M
    \frac{q_l}{r_{jl}}
  =
  R(0) + \sideset{}{^\prime}\sum_{l=1}^M
    q_l \left( \frac{1}{r_{jl}} - R(r_{jl}) \right)
  - \sum_{l=1}^M
    q_l R(r_{jl})
  \,,\quad j=1,\hdots,M
\end{equation*}



\begin{equation*}
  R(r)
  \approx
    \sum_{k=-N/2}^{N/2-1} \hat R_k \eim{kr}
\end{equation*}

\begin{equation*}
  \sum_{l=1}^M
    q_l R(r_{jl})
  \approx
  \sum_{l=1}^M q_l
    \sum_{k=-N/2}^{N/2-1} \hat R_k \eim{kr_{jl}}
  =
  \sum_{k=-N/2}^{N/2-1} \hat R_k
    \left(
      \sum_{l=1}^M q_l \eip{kr_l}
    \right)
    \eim{kr_j}
\end{equation*}

Near field interactions are computed with the ScaFaCoS near field solver, but can also be redirected to any other the near field solver.

Far field interactions are computed via convolution in Fourier space.
With the appropriate choice of window functions and convolutions in Fourier space,
\ptwonfft becomes the P3M (default for fully periodic boundaries), the NFFT based fast summation (default for fully non-periodic boundaries)
or almost any other FFT based fast Coulomb solver like PME, SPME, Gaussian split Ewald or plain Ewald summation.
Note, that an uniform particle distribution is required in order to achieve the typical $\mathcal{O}(N\log N)$ scaling.

\subsection{Calculation of the Electrostatic Fields}

\subsection{Calculation of the Virial}






\newpage
\section{Features}

\begin{description}
\item[Periodicity:] Only fully periodic and fully non-periodic boundaries are supported.
\item[Box shape:] Only cubic box shape is supported.
\item[Autotuning:] For fully periodic boundaries, the parameters can be automatically tuned when a
  required tolerance in the absolute rms field error is provided. For fully non-periodic boundaries,
  the parameters are more or less intelligently guessed when a required tolerance in the absolute
  potential error is provided.  
\item[Delegate near-field:] Yes.
\item[Virial:] Yes for fully non-periodic boundaries. For fully periodic boundaries we only compute an approximation of
  the diagonal of the virial tensor, \ie, all entries of the diagonal are set equal to one third of the total energy.
\end{description}

\section{Solver-specific Parameters}
The tolerance parameters are common to all ScaFaCoS solvers. At the moment \ptwonfft only supports automatic parameter tuning for the following tolerance types.
\begin{itemize}
  \item \verb!tolerance_type! -
    The type of error that we want to check for. Allowed values are \verb!FCS_TOLERANCE_TYPE_FIELD! (absolute rms error in the fields)
    for fully periodic boundaries and \verb!FCS_TOLERANCE_TYPE_POTENTIAL! (absolute rms error in the potentials) for non-periodic boundaries.
  \item \verb!tolerance! -
    The allowed tolerance of the error. Type of the error is given by \verb!tolerance_type!. If the required tolerance can't be achieved
    with the given parameters, \ptwonfft aborts with an error message.
\end{itemize}

In addition, \ptwonfft offers the following parameters, which can be adjusted by the \ptwonfft-specific functions described in the next section.
Alternatively, you can use the ScaFaCoS test program together with the optional command line argument \verb!-c! and a comma separated list
of parameter settings, e.g., use
\begin{verbatim}
./scafacos_test p2nfft systems/3d-periodic/cloud_wall.xml.gz \
-c tolerance_field,1e-4,p2nfft_r_cut,4.5,pnfft_N,16,16,16,pfft_patience,0
\end{verbatim}
in order to run the test program with a near field cut off range of $4.5$ and a sloppy planned FFT of size $16\times 16 \times 16$ up to tolerance $10^{-4}$ in the fields.
\begin{itemize}
  \item \verb!p2nfft_r_cut! -
    Absolute cutoff distance of the near field. Can be automatically tuned.
    Feasible values are any positive floats. Especially, \ptwonfft still works correct if \verb!p2nfft_r_cut! exceeds the simulation box.
    We do not require that \verb!p2nfft_r_cut! has to be less than half the smallest simulation box length.
    When \verb!p2nfft_r_cut! is chosen too small, the errors of the algorithm become large,
    the required accuracy might not be achieved, or the grid has to be chosen very large, which will slow down the algorithm.
    When \verb!p2nfft_r_cut! is chosen too large, the algorithm might become slow as most computation is done in the near field region.
    This parameter corresponds to \verb!r_cut! of the P3M solver.
  \item \verb!p2nfft_epsI! -
    Relative cutoff distance of the near field. \ptwonfft scales the original box size into a cube of side length 1.
    For fully periodic boundary conditions this corresponds to a simple scaling by the box size. However, the scaling is more complicated if non-periodic
    boundary conditions are involved. Application of the same scaling factor on \verb!r_cut! gives \verb!epsI!.
    Feasible values are any positive floats that fulfill $\verb!epsI! < 0.5$.
  \item \verb!p2nfft_grid! (acronym: \verb!pnfft_N!) -
    Size of the FFT grid. This is a acronym for \verb!pnfft_N! and is used in the same way. For details look at the PNFFT-specific parameters list.
    This parameter corresponds to the FFT grid size (\verb!p3m_grid!) of the P3M solver.
  \item \verb!p2nfft_oversampled_grid! (acronym: \verb!pnfft_n!) -
    Size of the oversampled FFT grid.
    This is a acronym for \verb!pnfft_n! and is used in the same way. For details look at the PNFFT-specific parameters list.
  \item \verb!p2nfft_alpha! -
    Ewald splitting parameter $\alpha$. Can be automatically tuned. This parameter balances the errors of the Ewald splitting corresponding to
    the real space and Fourier space contributions. Choosing $alpha$ too small (too large) means that the real space (Fourier space) error becomes dominating.
    This parameter corresponds to \verb!alpha! of the P3M solver.
  \item \verb!p2nfft_cao! (see also: \verb!pnfft_m!)-
    Charge assignment order. Can be automatically tuned. This parameter mostly corresponds to \verb!cao! of the P3M solver.
    However, note that this is only a wrapper that sets \verb!pnfft_m! to the value \verb!(p2nfft_cao+1)/2! (division rounded down).
    Therefore, \ptwonfft will always use an even charge assignment order.
  \item \verb!p2nfft_reg_near! -
    Choose between the two near field regularization methods for non-periodic boundaries. Feasible values are $0$ (Fourier coefficients have been precomputed via CG iteration, only available for some sets of parameters)
    and $1$ (Regularization with two-point Taylor polynomials). Default value is $0$, if possible.
  \item \verb!p2nfft_reg_near_name! (alternative: \verb!p2nfft_reg_near!) -
    Choose between the two regularization methods for non-periodic boundaries. Feasible values are \verb!cg! (Fourier coefficients have been precomputed via CG iteration, only available for some sets of parameters)
    and \verb!t2p! (Regularization with two-point Taylor polynomials). Default is \verb!cg!, if possible.
  \item \verb!p2nfft_reg_far! -
    Choose between the available regularization methods at the far field boundary for non-periodic boundaries.
    Feasible values are 
    $0$ (Fourier coefficients have been precomputed via CG iteration, only available for some sets of parameters and only available in combination with the flag \verb!rad_cg! for near field regularization),
    $1$ (radial regularization with symmetric two-point Taylor polynomials),
    $2$ (radial regularization with two-point Taylor polynomials and explicitly chosen constant continuation value; see \verb!p2nfft_c!),
    $3$ (radial regularization with two-point Taylor polynomials and implicitly chosen constant continuation value),
    $4$ (rectangular regularization with symmetric two-point Taylor polynomials),
    $5$ (rectangular regularization with two-point Taylor polynomials and explicitly chosen constant continuation value; see \verb!p2nfft_c!), and
    $6$ (rectangular regularization with two-point Taylor polynomials and implicitly chosen constant continuation value). 
    Default value is $3$.
  \item \verb!p2nfft_reg_far_name! (alternative: \verb!p2nfft_reg_far!) -
    Choose between the two regularization methods for non-periodic boundaries. Feasible values are
    \verb!rad_cg!      (radial regularization based on the \verb!cg! near field regularization, only available for some sets of parameters),
    \verb!rad_t2p_sym! (radial regularization with symmetric two-point Taylor polynomial),
    \verb!rad_t2p_ec!  (radial regularization with two-point Taylor polynomial and explicitly chosen constant continuation value; see \verb!p2nfft_c!),
    \verb!rad_t2p_ic!  (radial regularization with two-point Taylor polynomial and implicitly chosen constant continuation value),
    \verb!rec_t2p_sym! (rectangular regularization with symmetric two-point Taylor polynomial),
    \verb!rec_t2p_ec!  (rectangular regularization with two-point Taylor polynomial and explicitly chosen constant continuation value; see \verb!p2nfft_c!), and
    \verb!rec_t2p_ic!  (rectangular regularization with two-point Taylor polynomial and implicitly chosen constant continuation value).
    Default value is \verb!rad_t2p_ic!.
  \item \verb!p2nfft_epsB! -
    Regularization border for \ptwonfft with non-periodic boundary conditions. This parameter corresponds to the rescaled system into a unit cube of side length 1.
    Feasible values are any positive floats that fulfill $\verb!epsI! + \verb!epsB! < 0.5$.
    This parameter does not have any effect for fully periodic boundary conditions. Default value is $\verb!epsB! := \verb!epsI!$.
  \item \verb!p2nfft_p! -
    The degree of the two-point Taylor polynomial. This parameter only takes affect, if regularization \verb!t2p! has been enabled.
    Feasible values are any integers between $1$ and $16$. Default value is $7$.
  \item \verb!p2nfft_c! -
    The constant continuation value of the far field regularization. This parameter only takes affect, if far field regularization \verb!rad_t2p_ec! or \verb!rec_t2p_ec! is enabled.
    Feasible values are any floating point numbers. Default value is $0.0$.
  \item \verb!p2nfft_require_virial! -
    Decide if the computation of the virial should be included in \ptwonfft.
    Feasible values are $0$ (disable virial computation), or any other integer (enable virial computation).
    Default value is $0$ (disable virial computation).
  \item \verb!p2nfft_intpol_order! -
    \ptwonfft uses interpolation tables to speed up the repeated computation of the near field correction. The order of interpolation is given by \verb!intpol_order!.
    Feasible values are $-2$ (approximation of the error function according to \cite[eq.~(7.1.26)]{AbSt72}, which yields an error of $1.5\times 10^{-7}$),
    $-1$ (direct evaluation of the error function), $0$ (constant interpolation), $1$ (linear interpolation), $2$ (quadratic interpolation) and $3$ (cubic interpolation).
    Default value is $3$.
  \item \verb!p2nfft_k_cut! -
    Spherical cutoff distance in Fourier space. If this parameter is set to a positive float $k_c>0$, all Fourier coefficients $\hat f_{\mathbf k}$ with
    $\|\mathbf k\|_2 > kc$ will be set to zero, i.e., the approximation in Fourier space uses a spherical cutoff scheme.
    This is useful, e.g., for comparison of results to classical Ewald summation.
    Feasible values are any positive float (enable spherical cutoff) or any negative float (disable spherical cutoff).
    The spherical cutoff is not applied per default, since FFTs naturally use a box shaped cutoff scheme.
  \item \verb!p2nfft_ignore_tolerance! -
    On default, \ptwonfft aborts with an error message, if the required error tolerance can not be reached.
    This flag disables the check for accuracy and gives the possibility to run \ptwonfft
    with any set of parameters. It is intended for very experienced users that tune all parameters manually.
    Default value is $0$ (abort if tolerance check fails). Any other value disables the tolerance check.
  \item \verb!p2nfft_ignore_field! -
    Disable the computation of the field during \verb!fcs_run!.
    Default value is $0$ (compute field). Any other integer value disables the field computation.
  \item \verb!p2nfft_verbose_tuning! -
    Decide if a full list of \ptwonfft parameter settings is printed on stdout or not.
    Feasible values are $0$ (do not print), or any other integer (print the parameters).
    The default value is $1$ if \project was configured with the \verb+--enable-fcs-info+ flag, and $0$ otherwise.
\end{itemize}

\subsection{PNFFT-specific Parameters}
\begin{itemize}
  \item \verb!pnfft_N! (acronym: \verb!p2nfft_grid!) -
    Size of the FFT grid. Can be automatically tuned.
    Feasible values are three comma separated, even, positive integers, e.g., \verb!pnfft_N,16,32,12!.
    Optimal values are in the order of $M^{1/3}$, \ie, one grid point for each particle.
    The grid size may be any even, positive integer, but powers of
    two are recommended. The larger the grid, the smaller the
    error, but also the higher the memory and computational requirements
    of the algorithm.
    This parameter corresponds to the FFT grid size (\verb!grid!) of the P3M solver.
  \item \verb!pnfft_n! (acronym: \verb!p2nfft_oversampled_grid!) -
    Size of the oversampled FFT grid. Can be automatically tuned.
    Especially for non-periodic boundaries it is necessary to introduce oversampling to reduce the aliasing error.
    Feasible values are three comma separated, even integers with $\verb!n! \ge \verb!N!$ in every dimension, e.g., \verb!pnfft_N,16,32,12,pnfft_n,20,64,12!.
    Default value is \verb!N! for fully periodic boundaries and 2\verb!N! for fully non-periodic
    boundaries.
  \item \verb!pnfft_window! (alternative: \verb!pnfft_window_name!) -
    NFFT window function. Feasible values are $0$ (Kaiser-Bessel window), $1$ (Gaussian window),
    $2$ (B-spline window), $3$ (Sine Cardinal window - Fourier transform of the B-spline window) and $4$
    (window function based on the modified Bessel function of first kind - Fourier transform of the Kaiser-Bessel window).
    Default value is $2$ (B-spline). Recommended values are $0$ (Kaiser-Bessel) for fully non-periodic boundaries and $2$ (B-spline) for fully periodic boundaries.
  \item \verb!pnfft_window_name! (alternative: \verb!pnfft_window!) -
    Name of the NFFT window function. Feasible values are \verb!kaiser! (Kaiser-Bessel window), \verb!gaussian! (Gaussian window),
    \verb!bspline! (B-spline window), \verb!sinc! (Sine Cardinal window - Fourier transform of the B-spline window) and \verb!bessel_i0!
    (window function based on the modified Bessel function of first kind - Fourier transform of the Kaiser-Bessel window).
    Default value is \verb!bspline!. Recommended values are \verb!kaiser! for fully non-periodic boundaries and \verb!bspline! for fully periodic boundaries.
  \item \verb!pnfft_m! (see also: \verb!p2nfft_cao!) -
    Real space cutoff of the window function. The number of grid points that are influenced
    by a charged particle. Can be automatically tuned. Allowed values are any positive integers.
    In most cases \verb!m! is chosen between $1$ (low precision) and $16$ (very high precision).\\
    {\bfseries Note: Due to historical reasons, this parameter corresponds to one half of the charge assignment order
    (\verb!cao!) of the P3M solver!}
  \item \verb!pnfft_b! -
    Shape parameter of the NFFT window function corresponding to the three dimensions. Can be automatically tuned.
    Feasible values are three comma separated positive floating point numbers, e.g., \verb!pnfft_b,0.5,1.0,3.2!.
    Choosing an optimal value for b is not that easy but influences the error a lot.
    Therefore, you should only change the default value if you know what you do.
  \item \verb!pnfft_intpol_order! -
    PNFFT uses interpolation tables to speed up the repeated evaluation of the window functions. The integer \verb!pnfft_intpol_order! gives the interpolation order.
    Feasible values are $-1$ (direct evaluation of window function without interpolation), $0$, (constant interpolation), $1$ (linear interpolation), $2$ (quadratic interpolation) and $3$ (cubic interpolation).
    Default value is $3$ (cubic interpolation).
  \item \verb!pnfft_direct! -
    Flag to enable plain NDFT instead of the fast approximate NFFT algorithms. 
    Feasible values are $0$ (use NFFT), or any other integer \verb!pnfft_direct!$\ne0$ (use NDFT). Default value is \verb!pnfft_direct!$=0$ (use NFFT).
  \item \verb!pnfft_pre_phi_hat! -
    Flag to enable precomputed Fourier coefficients for faster computation of the diagonal matrix in the NFFT algorithm.
    Feasible values are $0$ (turn off pre-computation), or any other integer \verb!pnfft_pre_phi_hat!$\ne0$ (enable pre-computation).
    Default value is $1$ (enable pre-computation).
  \item \verb!pnfft_fast_gaussian! -
    Set PNFFT flag \verb!PNFFT_FAST_GAUSSIAN!. Only Gaussian window function we can use Fast Gaussian Gridding in order to speed up the evaluation of the window function.
    Feasible values are $0$ (switch Fast Gaussian Gridding off), or any other integer \verb!pnfft_fast_gaussian!$\ne0$ (use Fast Gaussian Gridding).
    Default value is $0$ (since B-Spline is the default window function).
  \item \verb!pnfft_fft_in_place! -
    Set PNFFT flag \verb!PNFFT_FFT_IN_PLACE!. This causes the PNFFT planner to use in-place-FFTs, which saves about half of the memory for FFT but may sacrifices some performance.
    Feasible values are $0$ (call out-of-place FFTs), or any other integer \verb!pnfft_fft_in_place!$\ne0$ (call in-place-FFTs).
    Default value is \verb!pnfft_fft_in_place!$=0$ (call out-of-place FFTs).
  \item \verb!pnfft_sort_nodes! -
    Set PNFFT flag \verb!PNFFT_SORT_NODES!. Chooses whether to call a local sort before the interpolation step. This may result in better cache locality.
    Feasible values are $0$ (omit local sort), or any other integer \verb!pnfft_sort_nodes!$\ne0$ (sort before interpolation).
    Default value is \verb!pnfft_sort_nodes!$=0$ (omit local sort).
  \item \verb!pnfft_interlaced! -
    Set PNFFT flag \verb!PNFFT_INTERLACED!. Chooses whether PNFFT uses interlacing. This gives better accuracy at the price of more operations.
    Feasible values are $0$ (omit interlacing), or any other integer \verb!pnfft_interlaced!$\ne0$ (enable interlacing).
    Default value is \verb!pnfft_interlaced!$=0$ (omit interlacing).
  \item \verb!pnfft_diff_ik! -
    Set PNFFT flag \verb!PNFFT_DIFF_IK!. Chooses whether PNFFT computes the gradient through multiplication with $-2\pi\textrm{i}\mathbf{k}$ in Fourier space.
    This gives better accuracy in comparison to analytic differentiation at the price of more operations. Especially, we have to call four backward FFTs (instead of one for analytic differentiation).
    Feasible values are $0$ (use analytic differentiation), or any other integer \verb!pnfft_diff_ik!$\ne0$ (use Fourier space differentiation).
    Default value is \verb!pnfft_diff_ik!$=0$ (use analytic differentiation).
%   \item \verb!pnfft_pre_psi! -
%     Set PNFFT flag \verb!PNFFT_PRE_PSI!. Tensor product based precomputation of the window function. This option reduces the amount of repeated evaluations of the window function at the cost of
%     $6(2m+1)M$ floats (or $3(2m+1)M$ floats if the gradient is not needed).
%     Since the this kind of precomputation depends on the nodes $\mathbf x_j$, it must be performed during \verb!fcs_run! instead of \verb!fcs_tune!.
%     Feasible values are $0$ (switch off precomputation), or any other integer \verb!pnfft_pre_psi!$\ne0$ (switch on precomputation).
%     Default value is $0$ (no precomputation). This flag can not be used together with \verb!pnfft_pre_full_psi!.
%   \item \verb!pnfft_pre_fg_psi! -
%     Set PNFFT flags \verb!PNFFT_FG_PSI! and \verb!PNFFT_PRE_PSI!, i.e., use Fast Gaussian Gridding for tensor product based precomputation.
%     Since the this kind of precomputation depends on the nodes $\mathbf x_j$, it must be performed during \verb!fcs_run! instead of \verb!fcs_tune!.
%     Feasible values are $0$ (switch off precomputation), or any other integer \verb!pnfft_pre_fg_psi!$\ne0$ (switch on precomputation).
%     Default value is $0$ (no precomputation). This flag can not be used together with \verb!pnfft_pre_full_psi!.
%   \item \verb!pnfft_pre_full_psi! -
%     Set PNFFT flag \verb!PNFFT_PRE_FULL_PSI!. Full precomputation of the window function. This option reduces the amount of repeated evaluations of the window function at the cost of
%     $4(2m+1)^3M$ floats (or $(2m+1)^3M$ floats if the gradient is not needed).
%     Since the this kind of precomputation depends on the nodes $\mathbf x_j$, it must be performed during \verb!fcs_run! instead of \verb!fcs_tune!.
%     Feasible values are $0$ (switch off precomputation), or any other integer \verb!pnfft_pre_psi!$\ne0$ (switch on precomputation).
%     Default value is $0$ (no precomputation). This flag can not be used together with \verb!pnfft_pre_full_psi!.
%   \item \verb!pnfft_pre_full_fg_psi! -
%     Set PNFFT flags \verb!PNFFT_FULL_FG_PSI! and \verb!PNFFT_PRE_FULL_PSI!, i.e., use Fast Gaussian Gridding for full window function precomputation.
%     Since the this kind of precomputation depends on the nodes $\mathbf x_j$, it must be performed during \verb!fcs_run! instead of \verb!fcs_tune!.
%     Feasible values are $0$ (switch off precomputation), or any other integer \verb!pnfft_pre_full_fg_psi!$\ne0$ (switch on precomputation).
%     Default value is $0$ (no precomputation). This flag can not be used together with \verb!pnfft_pre_psi!.
  \item \verb!pnfft_real_f! -
    Set PNFFT flag \verb!PNFFT_REAL_F!. Normally, PNFFT works on complex valued inputs. This flag enables some optimizations for real valued inputs, e.g., substituting complex additions and multiplications with real ones.
    However, this comes at the price of strided memory access and does not always improve performance.
    Feasible values are $0$ (use complex operations), or any other integer \verb!pnfft_real_f!$\ne0$ (use real operations where possible).
    Default value is \verb!pnfft_real_f!$=0$ (use complex operations).
\end{itemize}

\subsection{PFFT-specific Parameters}
\begin{itemize}
  \item \verb!pfft_patience! -
    Similar to FFTW, the PFFT library splits the computation of parallel FFT into a more or less time consuming planning step and a fast execution step.
    The time spent for planning can be adjusted by the \verb!pfft_patience! flag. Feasible values are $0$ (plan PFFT with \verb!PFFT_ESTIMATE!), $1$ (plan PFFT with \verb!PFFT_MEASURE!),
    $2$ (plan PFFT with \verb!PFFT_PATIENT!) and $3$ (plan PFFT with \verb!PFFT_EXHAUSTIVE!). All other values raise an error. Default value is $1$ (\verb!PFFT_MEASURE!).
  \item \verb!pfft_tune! -
    The PFFT library uses FFTW for computing serial FFTs combined with serial transpositions. Sometimes its better to perform the FFT and transposition in two separate steps,
    but the FFTW planner does not recognize. For value \verb!pfft_tune!$=0$ PFFT uses the FFTW plan as it is. For any other value, PFFT calls an additional planner in order to decide
    if performance gets better when the local FFT and transposition is performed in two separate steps. For large local array sizes this tuning gets very time consuming.
    Default value is $0$ (tuning turned off).
  \item \verb!pfft_preserve_input! -
    PFFT can chose between a larger set of algorithms, if it is allowed to overwrite the input array.
    Feasible values are $0$ (PFFT is allowed to destroy input), or any other integer \verb!pfft_preserve_input!$\ne0$ (PFFT is not allowed to destroy input).
    Default value is $0$ (PFFT is allowed to destroy input).
\end{itemize}


\section{Solver-specific Functions}
\begin{itemize}
  \item
\begin{alltt}
FCSResult fcs_p2nfft_set_r_cut(FCS handle, fcs_float r_cut);
FCSResult fcs_p2nfft_get_r_cut(FCS handle, fcs_float* r_cut);
FCSResult fcs_p2nfft_set_r_cut_tune(FCS handle);
\end{alltt}
    Set/retrieve/restore spherical near field cutoff radius.
  \item
\begin{alltt}
FCSResult fcs_p2nfft_set_epsI(FCS handle, fcs_float eps_I);
FCSResult fcs_p2nfft_get_epsI(FCS handle, fcs_float* eps_I);
FCSResult fcs_p2nfft_set_epsI_tune(FCS handle);
\end{alltt}
    Set/retrieve/restore normalized near field cutoff radius.
  \item
\begin{alltt}
FCSResult fcs_p2nfft_set_grid(
    FCS handle, fcs_int N0, fcs_int N1, fcs_int N2);
FCSResult fcs_p2nfft_get_grid(
    FCS handle, fcs_int* N0, fcs_int* N1, fcs_int* N2);
FCSResult fcs_p2nfft_set_grid_tune(FCS handle);
\end{alltt}
    Set/retrieve/restore FFT grid size.
  \item
\begin{alltt}
FCSResult fcs_p2nfft_set_oversampled_grid(
    FCS handle, fcs_int n0, fcs_int n1, fcs_int n2);
FCSResult fcs_p2nfft_get_oversampled_grid(
    FCS handle, fcs_int* n0, fcs_int* n1, fcs_int* n2);
FCSResult fcs_p2nfft_set_oversampled_grid_tune(FCS handle);
\end{alltt}
    Set/retrieve/restore oversampled FFT grid size.
  \item
\begin{alltt}
FCSResult fcs_p2nfft_set_alpha(FCS handle, fcs_float alpha);
FCSResult fcs_p2nfft_get_alpha(FCS handle, fcs_float* alpha);
FCSResult fcs_p2nfft_set_alpha_tune(FCS handle);
\end{alltt}
    Set/retrieve/restore Ewald splitting parameter.
  \item
\begin{alltt}
FCSResult fcs_p2nfft_set_cao(FCS handle, fcs_int cao);
FCSResult fcs_p2nfft_get_cao(FCS handle, fcs_int* cao);
FCSResult fcs_p2nfft_set_cao_tune(FCS handle);
\end{alltt}
    Set/retrieve/restore charge assignment order.
  \item
\begin{alltt}
FCSResult fcs_p2nfft_set_reg_near(FCS handle, fcs_int reg);
FCSResult fcs_p2nfft_set_reg_near_by_name(
    FCS handle, char* reg_name);
FCSResult fcs_p2nfft_get_reg_near(FCS handle, fcs_int* reg);
\end{alltt}
    Set by number/set by name/retrieve number of near field regularization (default = 0).
    Feasible values are
    0 (\verb!"cg"!), and
    1 (\verb!"t2p"!).
  \item
\begin{alltt}
FCSResult fcs_p2nfft_set_reg_far(FCS handle, fcs_int reg);
FCSResult fcs_p2nfft_set_reg_far_by_name(
    FCS handle, char* reg_name);
FCSResult fcs_p2nfft_get_reg_far(FCS handle, fcs_int* reg);
\end{alltt}
    Set by number/set by name/retrieve number of near field regularization (default = 0).
    Feasible values are
    0 (\verb!"rad_cg"!),
    1 (\verb!"rad_t2p_sym"!),
    2 (\verb!"rad_t2p_ec"!),
    3 (\verb!"rad_t2p_ic"!),
    4 (\verb!"rec_t2p_sym"!),
    5 (\verb!"rec_t2p_ec"!), and 
    6 (\verb!"rec_t2p_ic"!).
  \item
\begin{alltt}
FCSResult fcs_p2nfft_set_epsB(FCS handle, fcs_float eps_B);
FCSResult fcs_p2nfft_get_epsB(FCS handle, fcs_float* eps_B);
FCSResult fcs_p2nfft_set_epsB_tune(FCS handle);
\end{alltt}
    Set/retrieve/restore normalized far field regularization border.
  \item
\begin{alltt}
FCSResult fcs_p2nfft_set_p(FCS handle, fcs_int p);
FCSResult fcs_p2nfft_get_p(FCS handle, fcs_int* p);
FCSResult fcs_p2nfft_set_p_tune(FCS handle);
\end{alltt}
    Set/retrieve/restore polynomial degree of two-point Taylor regularization.
  \item
\begin{alltt}
FCSResult fcs_p2nfft_set_c(FCS handle, fcs_float c);
FCSResult fcs_p2nfft_get_c(FCS handle, fcs_float* c);
FCSResult fcs_p2nfft_set_c_tune(FCS handle);
\end{alltt}
    Set/retrieve/restore explicit continuation value for two-point Taylor regularization.
  \item
\begin{alltt}
FCSResult fcs_p2nfft_require_virial(
    FCS handle, fcs_int require_virial);
\end{alltt}
    Enable virial computation (optional, default = 0).
\begin{alltt}
FCSResult fcs_p2nfft_get_virial(FCS handle, fcs_float* virial);
\end{alltt}
    Retrieve virial (optional, default)
\begin{alltt}
FCSResult fcs_p2nfft_virial_is_active(
    FCS handle, fcs_int* yes_or_no);
\end{alltt}
    Check if virial computation is enabled.
  \item
\begin{alltt}
FCSResult fcs_p2nfft_set_interpolation_order(
    FCS handle, fcs_int intpol_order);
FCSResult fcs_p2nfft_get_interpolation_order(
    FCS handle, fcs_int* intpol_order);
\end{alltt}
    Set/retrieve interpolation order of near field correction (optional, default = 3).
  \item
\begin{alltt}
FCSResult fcs_p2nfft_set_k_cut(FCS handle, fcs_float k_cut);
FCSResult fcs_p2nfft_get_k_cut(FCS handle, fcs_float* k_cut);
FCSResult fcs_p2nfft_set_k_cut_tune(FCS handle);
\end{alltt}
    Set/retrieve/restore spherical far field cutoff radius.
  \item
\begin{alltt}
FCSResult fcs_p2nfft_set_ignore_tolerance(
    FCS handle, fcs_int set_ignore_tolerance);
FCSResult fcs_p2nfft_get_ignore_tolerance(
    FCS handle, fcs_int* set_ignore_tolerance);
\end{alltt}
    Set/retrieve flag for disabling the accuracy check (default = 0).
  \item
\begin{alltt}
FCSResult fcs_p2nfft_set_ignore_field(
    FCS handle, fcs_int set_ignore_field);
FCSResult fcs_p2nfft_get_ignore_field(
    FCS handle, fcs_int* set_ignore_field);
\end{alltt}
    Set/retrieve flag for disabling the field computation (default = 0).
  \item
\begin{alltt}
FCSResult fcs_p2nfft_set_verbose_tuning(
    FCS handle, fcs_int set_verbose_tuning);
FCSResult fcs_p2nfft_get_verbose_tuning(
    FCS handle, fcs_int* set_verbose_tuning);
\end{alltt}
    Set/retrieve flag for verbose tuning. The default value will be 1 (library was configured with \verb!--enable-fcs-info!) or 0 (otherwise).
\end{itemize}

\subsection{PNFFT-specific Functions}
\begin{itemize}
  \item
\begin{alltt}
FCSResult fcs_p2nfft_set_pnfft_N(
    FCS handle, fcs_int N0, fcs_int N1, fcs_int N2);
FCSResult fcs_p2nfft_get_pnfft_N(
    FCS handle, fcs_int* N0, fcs_int* N1, fcs_int* N2);
FCSResult fcs_p2nfft_set_pnfft_N_tune(FCS handle);
\end{alltt}
    Set/retrieve/restore FFT grid size.
  \item
\begin{alltt}
FCSResult fcs_p2nfft_set_pnfft_n(
    FCS handle, fcs_int n0, fcs_int n1, fcs_int n2);
FCSResult fcs_p2nfft_get_pnfft_n(
    FCS handle, fcs_int* n0, fcs_int* n1, fcs_int* n2);
FCSResult fcs_p2nfft_set_pnfft_n_tune(FCS handle);
\end{alltt}
    Set/retrieve/restore oversampled FFT grid size.
  \item
\begin{alltt}
FCSResult fcs_p2nfft_set_pnfft_window(
    FCS handle, fcs_int window);
FCSResult fcs_p2nfft_get_pnfft_window(
    FCS handle, fcs_int* window);
FCSResult fcs_p2nfft_set_pnfft_window_by_name(
    FCS handle, char* window_name );
\end{alltt}
    Set/retrieve \verb!pnfft_window! by number or name. (optional, default = 1).
    Feasible values are 0 (\verb!"gaussian"!), 1 (\verb!"bspline"!), 2 (\verb!"sinc"!), 3 (\verb!"kaiser"!) and 4 (\verb!"bessel_i0"!).
  \item
\begin{alltt}
FCSResult fcs_p2nfft_set_pnfft_m(FCS handle, fcs_int m);
FCSResult fcs_p2nfft_get_pnfft_m(FCS handle, fcs_int* m);
FCSResult fcs_p2nfft_set_pnfft_m_tune(FCS handle);
\end{alltt}
    Set/retrieve/restore real space cutoff of window function.
  \item
\begin{alltt}
FCSResult fcs_p2nfft_set_pnfft_b(FCS handle, fcs_float b0, fcs_float b1, fcs_float b2);
FCSResult fcs_p2nfft_get_pnfft_b(FCS handle, fcs_float* b0, fcs_float* b1, fcs_float* b2);
FCSResult fcs_p2nfft_set_pnfft_b_tune(FCS handle);
\end{alltt}
    Set/retrieve/restore shape parameters of window function corresponding to each of the three dimensions.
  \item
\begin{alltt}
FCSResult fcs_p2nfft_set_pnfft_interpolation_order(
    FCS handle, fcs_int intpol_order);
FCSResult fcs_p2nfft_get_pnfft_interpolation_order(
    FCS handle, fcs_int* intpol_order);
\end{alltt}
    Set/retrieve \verb!pnfft_intpol_order! (optional, default = 3).
  \item
\begin{alltt}
FCSResult fcs_p2nfft_set_pnfft_direct(
    FCS handle, fcs_int yes_or_no);
FCSResult fcs_p2nfft_get_pnfft_direct(
    FCS handle, fcs_int* yes_or_no);
\end{alltt}
    Set/retrieve flag that determines if plain NDFT is used instead of NFFT.
    (optional, default = 0).
  \item
\begin{alltt}
FCSResult fcs_p2nfft_set_pnfft_pre_phi_hat(
    FCS handle, fcs_int yes_or_no);
FCSResult fcs_p2nfft_get_pnfft_pre_phi_hat(
    FCS handle, fcs_int* yes_or_no);
\end{alltt}
    Set/retrieve flag \verb!pnfft_pre_phi_hat! (optional, default = 0).
  \item
\begin{alltt}
FCSResult fcs_p2nfft_set_pnfft_fast_gaussian(
    FCS handle, fcs_int yes_or_no);
FCSResult fcs_p2nfft_get_pnfft_fast_gaussian(
    FCS handle, fcs_int* yes_or_no);
\end{alltt}
    Set/retrieve flag \verb!pnfft_fast_gaussian! (optional, default = 0).
  \item
\begin{alltt}
FCSResult fcs_p2nfft_set_pnfft_fft_in_place(
    FCS handle, fcs_int yes_or_no);
FCSResult fcs_p2nfft_get_pnfft_fft_in_place(
    FCS handle, fcs_int* yes_or_no);
\end{alltt}
    Set/retrieve flag \verb!pnfft_fft_in_place! (optional, default = 0).
  \item
\begin{alltt}
FCSResult fcs_p2nfft_set_pnfft_sort_nodes(
    FCS handle, fcs_int yes_or_no);
FCSResult fcs_p2nfft_get_pnfft_sort_nodes(
    FCS handle, fcs_int* yes_or_no);
\end{alltt}
    Set/retrieve flag \verb!pnfft_sort_nodes! (optional, default = 0).
  \item
\begin{alltt}
FCSResult fcs_p2nfft_set_pnfft_interlaced(
    FCS handle, fcs_int yes_or_no);
FCSResult fcs_p2nfft_get_pnfft_interlaced(
    FCS handle, fcs_int* yes_or_no);
\end{alltt}
    Set/retrieve flag \verb!pnfft_interlaced! (optional, default = 0).
  \item
\begin{alltt}
FCSResult fcs_p2nfft_set_pnfft_diff_ik(
    FCS handle, fcs_int yes_or_no);
FCSResult fcs_p2nfft_get_pnfft_diff_ik(
    FCS handle, fcs_int* yes_or_no);
\end{alltt}
    Set/retrieve flag \verb!pnfft_diff_ik! (optional, default = 0).
%   \item
% \begin{alltt}
% FCSResult fcs_p2nfft_set_pnfft_pre_psi(
%     FCS handle, fcs_int yes_or_no);
% FCSResult fcs_p2nfft_get_pnfft_pre_psi(
%     FCS handle, fcs_int* yes_or_no);
% \end{alltt}
%     Set/retrieve flag \verb!pnfft_pre_psi! (optional, default = 0).
%   \item
% \begin{alltt}
% FCSResult fcs_p2nfft_set_pnfft_pre_fg_psi(
%     FCS handle, fcs_int yes_or_no);
% FCSResult fcs_p2nfft_get_pnfft_pre_fg_psi(
%     FCS handle, fcs_int* yes_or_no);
% \end{alltt}
%     Set/retrieve flag \verb!pnfft_pre_fg_psi! (optional, default = 0).
%   \item
% \begin{alltt}
% FCSResult fcs_p2nfft_set_pnfft_pre_full_psi(
%     FCS handle, fcs_int yes_or_no);
% FCSResult fcs_p2nfft_get_pnfft_pre_full_psi(
%     FCS handle, fcs_int* yes_or_no);
% \end{alltt}
%     Set/retrieve flag \verb!pnfft_pre_full_psi! (optional, default = 0).
%   \item
% \begin{alltt}
% FCSResult fcs_p2nfft_set_pnfft_pre_full_fg_psi(
%     FCS handle, fcs_int yes_or_no);
% FCSResult fcs_p2nfft_get_pnfft_pre_full_fg_psi(
%     FCS handle, fcs_int* yes_or_no);
% \end{alltt}
%     Set/retrieve flag \verb!pnfft_pre_full_fg_psi! (optional, default = 0).
  \item
\begin{alltt}
FCSResult fcs_p2nfft_set_pnfft_real_f(
    FCS handle, fcs_int yes_or_no);
FCSResult fcs_p2nfft_get_pnfft_real_f(
    FCS handle, fcs_int* yes_or_no);
\end{alltt}
    Set/retrieve flag \verb!pnfft_real_f! (optional, default = 0).
\end{itemize}

\subsection{PFFT-specific Functions}
\begin{itemize}
  \item
\begin{alltt}
FCSResult fcs_p2nfft_set_pfft_patience(
    FCS handle, fcs_int pfft_patience_flag);
FCSResult fcs_p2nfft_get_pfft_patience(
    FCS handle, fcs_int* pfft_patience_flag);
FCSResult fcs_p2nfft_set_pfft_patience_by_name(
    FCS handle, char* patience_name );
\end{alltt}
    Set/retrieve flag \verb!pfft_patience! by number or name (optional, default = 1).
    Feasible values are 0 (\verb!estimate!), 1 (\verb!measure!), 2 (\verb!patient!) and 3 (\verb!exhaustive!).
  \item
\begin{alltt}
FCSResult fcs_p2nfft_set_pfft_tune(
    FCS handle, fcs_int yes_or_no);
FCSResult fcs_p2nfft_get_pfft_tune(
    FCS handle, fcs_int* yes_or_no);
\end{alltt}
    Set/retrieve flag \verb!pfft_tune! (optional, default = 0).
  \item
\begin{alltt}
FCSResult fcs_p2nfft_set_pfft_preserve_input(
    FCS handle, fcs_int yes_or_no);
FCSResult fcs_p2nfft_get_pfft_preserve_input(
    FCS handle, fcs_int* yes_or_no);
\end{alltt}
    Set/retrieve flag \verb!pfft_preserve_input! (optional, default = 0).
\end{itemize}





