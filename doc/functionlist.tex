\chapter{List of Functions}
\label{cha:appendix_function_list}

This chapter lists all the functions within the ScaFaCoS interface, which can be used by the user to manipulate the library.

\section{Mandatory Functions}
\begin{figure}[htb]
\begin{lstlisting}[language=C,frame=trBL,breaklines,basicstyle=\ttfamily]
/* C */
FCSResult fcs_init(FCS handle, char* method, MPI_Comm comm);
\end{lstlisting}
\begin{lstlisting}[language=Fortran,frame=trBL,breaklines,basicstyle=\ttfamily]
! Fortran
function fcs_init(handle,method,comm)
type(c_ptr)                                 :: handle
char                                        :: method(*)
integer(kind = c_int)                       :: comm
type(c_ptr)                                 :: fcs_init
\end{lstlisting}
\caption{Function call: fcs\_init}
\label{fig:fcs_init}
\ufunctiontoindex{fcs\_init}
\end{figure}
\begin{figure}[htb]
\begin{lstlisting}[language=C,frame=trBL,breaklines,basicstyle=\ttfamily,prebreak={\raisebox{0ex}[0ex][0ex]{\ensuremath{\hookleftarrow}}}]
/* C */
FCSResult fcs_set_common(FCS handle, fcs_int short_range_flag, fcs_float* box_a, fcs_float* box_b, fcs_float* box_c, fcs_float* offset, fcs_int* periodicity, fcs_int total_particles);
\end{lstlisting}
\begin{lstlisting}[language=Fortran,frame=trBL,breaklines,basicstyle=\ttfamily,prebreak={\raisebox{0ex}[0ex][0ex]{\ensuremath{\hookleftarrow}}}]
! Fortran
function fcs_set_common(handle,short_range_flag,box_a,box_b,box_c,offset,periodicity,total_particles)
type(c_ptr)                                 ::  handle
logical                                     ::  short_range_flag
real(kind = fcs_real_kind_isoc)             ::  box_a(3)
real(kind = fcs_real_kind_isoc)             ::  box_b(3)
real(kind = fcs_real_kind_isoc)             ::  box_c(3)
real(kind = fcs_real_kind_isoc)             ::  offset(3)
logical                                     ::  periodicity(3)
integer(kind = fcs_integer_kind_isoc)       ::  total_parts
type(c_ptr)                                 ::  fcs_set_common
\end{lstlisting}
\caption{Function call: fcs\_common\_set}
\label{fig:fcs_set_common}
\ufunctiontoindex{fcs\_common\_set}
\end{figure}
\begin{figure}[htb]
\begin{lstlisting}[language=C,frame=trBL,breaklines,basicstyle=\ttfamily,prebreak={\raisebox{0ex}[0ex][0ex]{\ensuremath{\hookleftarrow}}}]
/* C */
FCSResult fcs_tune (FCS handle, fcs_int local_particles, fcs_int local_max_particles, fcs_float *positions, fcs_float *charges);
\end{lstlisting}
\begin{lstlisting}[language=Fortran,frame=trBL,breaklines,basicstyle=\ttfamily,prebreak={\raisebox{0ex}[0ex][0ex]{\ensuremath{\hookleftarrow}}}]
! Fortran
function fcs_tune(handle,n_locp,n_maxlocp,positions,charges)
type(c_ptr), value                          :: handle
integer(kind = fcs_integer_kind_isoc),value :: n_locp
integer(kind = fcs_integer_kind_isoc),value :: n_maxlocp
real(kind = fcs_real_kind_isoc)             :: positions(3*n_locp)
real(kind = fcs_real_kind_isoc)             :: charges(n_locp)
type(c_ptr)                                 :: fcs_tune
\end{lstlisting}
\caption{Function call: fcs\_tune}
\label{fig:fcs_tune}
\ufunctiontoindex{fcs\_tune}
\end{figure}
\begin{figure}[htb]
\begin{lstlisting}[language=C,frame=trBL,breaklines,basicstyle=\ttfamily,prebreak={\raisebox{0ex}[0ex][0ex]{\ensuremath{\hookleftarrow}}}]
/* C */
FCSResult fcs_run (FCS handle, fcs_int local_particles, fcs_int local_max_particles, fcs_float *positions, fcs_float *charges, fcs_float *field, fcs_float *potentials);
\end{lstlisting}
\begin{lstlisting}[language=Fortran,frame=trBL,breaklines,basicstyle=\ttfamily,prebreak={\raisebox{0ex}[0ex][0ex]{\ensuremath{\hookleftarrow}}}]
! Fortran
function fcs_run(handle,n_locp,n_maxlocp,positions,charges,field,potential)
type(c_ptr), value                          :: handle
integer(kind = fcs_integer_kind_isoc),value :: n_locp
integer(kind = fcs_integer_kind_isoc),value :: n_maxlocp
real(kind = fcs_real_kind_isoc)             :: positions(3*n_locp)
real(kind = fcs_real_kind_isoc)             :: charges(n_locp)
real(kind = fcs_real_kind_isoc)             :: field(3*n_locp)
real(kind = fcs_real_kind_isoc)             :: potential(n_locp)
type(c_ptr)                                 :: fcs_run
\end{lstlisting}
\caption{Function call: fcs\_run}
\label{fig:fcs_run}
\ufunctiontoindex{fcs\_run}
\end{figure}
\begin{figure}[htb]
\begin{lstlisting}[language=C,frame=trBL,breaklines,basicstyle=\ttfamily,prebreak={\raisebox{0ex}[0ex][0ex]{\ensuremath{\hookleftarrow}}}]
/* C */
FCSResult fcs_destroy (FCS handle);
\end{lstlisting}
\begin{lstlisting}[language=Fortran,frame=trBL,breaklines,basicstyle=\ttfamily,prebreak={\raisebox{0ex}[0ex][0ex]{\ensuremath{\hookleftarrow}}}]
! Fortran
function fcs_destroy(handle)
type(c_ptr), value                          :: handle
type(c_ptr)                                 :: fcs_run
\end{lstlisting}
\caption{Function call: fcs\_destroy}
\label{fig:fcs_destroy}
\ufunctiontoindex{fcs\_destroy}
\end{figure}
\FloatBarrier
\section{Generic Functions}

\FloatBarrier
\subsection{Errorhandling}

\begin{figure}[htb]
\begin{lstlisting}[language=C,frame=trBL,breaklines,basicstyle=\ttfamily,prebreak={\raisebox{0ex}[0ex][0ex]{\ensuremath{\hookleftarrow}}}]
/* C */
fcs_int fcsResult_getReturnCode (FCSResult err);
\end{lstlisting}
\begin{lstlisting}[language=Fortran,frame=trBL,breaklines,basicstyle=\ttfamily,prebreak={\raisebox{0ex}[0ex][0ex]{\ensuremath{\hookleftarrow}}}]
! Fortran
function fcsResult_getReturnCode(res)
type(c_ptr)                            ::  res
integer(kind = fcs_integer_kind_isoc)  ::  fcsResult_getReturnCode
\end{lstlisting}
\caption{Function call: fcsResult\_getReturnCode}
\label{fig:fcsResult_getReturnCode}
\ufunctiontoindex{fcsResult\_getReturnCode}
\end{figure}

\begin{figure}[htb]
\begin{lstlisting}[language=C,frame=trBL,breaklines,basicstyle=\ttfamily,prebreak={\raisebox{0ex}[0ex][0ex]{\ensuremath{\hookleftarrow}}}]
/* C */
const char* fcsResult_getErrorMessage (FCSResult err);
\end{lstlisting}
\begin{lstlisting}[language=Fortran,frame=trBL,breaklines,basicstyle=\ttfamily,prebreak={\raisebox{0ex}[0ex][0ex]{\ensuremath{\hookleftarrow}}}]
! Fortran
function fcsResult_getErrorMessage(res)
type(c_ptr)                                     ::  res
character(kind = c_char, len = MESSAGE_LENGTH)  ::  fcsResult_getErrorMessage
\end{lstlisting}
\caption{Function call: fcsResult\_getErrorMessage}
\label{fig:fcsResult_getErrorMessage}
\ufunctiontoindex{fcsResult\_getErrorMessage}
\end{figure}

\begin{figure}[htb]
\begin{lstlisting}[language=C,frame=trBL,breaklines,basicstyle=\ttfamily,prebreak={\raisebox{0ex}[0ex][0ex]{\ensuremath{\hookleftarrow}}}]
/* C */
const char* fcsResult_getErrorSource (FCSResult err);
\end{lstlisting}
\begin{lstlisting}[language=Fortran,frame=trBL,breaklines,basicstyle=\ttfamily,prebreak={\raisebox{0ex}[0ex][0ex]{\ensuremath{\hookleftarrow}}}]
! Fortran
function fcsResult_getErrorSource(res)
type(c_ptr)                                     ::  res
character(kind = c_char, len = MESSAGE_LENGTH)  ::  fcsResult_getErrorSource
\end{lstlisting}
\caption{Function call: fcsResult\_getErrorSource}
\label{fig:fcsResult_getErrorSource}
\ufunctiontoindex{fcsResult\_getErrorSource}
\end{figure}

\FloatBarrier
\subsection{Getters and Setters}



\FloatBarrier
\subsection{Near Field Computations}

\begin{figure}[htb]
\begin{lstlisting}[language=C,frame=trBL,breaklines,basicstyle=\ttfamily,prebreak={\raisebox{0ex}[0ex][0ex]{\ensuremath{\hookleftarrow}}}]
/* C */
FCSResult fcs_compute_near_field (FCS handle, fcs_float dist, fcs_float *field);
\end{lstlisting}
\begin{lstlisting}[language=Fortran,frame=trBL,breaklines,basicstyle=\ttfamily,prebreak={\raisebox{0ex}[0ex][0ex]{\ensuremath{\hookleftarrow}}}]
! Fortran
function fcs_compute_near_field(handle,dist,field)
type(c_ptr)                                     ::  handle
real(kind = fcs_real_kind_isoc)                 ::  dist
real(kind = fcs_real_kind_isoc), dimension(3)   ::  field
type(c_ptr)                                     ::  fcs_compute_near_field
\end{lstlisting}
\caption{Function call: fcs\_compute\_near\_field}
\label{fig:fcs_compute_near_field}
\ufunctiontoindex{fcs\_compute\_near\_field}
\end{figure}

\begin{figure}[htb]
\begin{lstlisting}[language=C,frame=trBL,breaklines,basicstyle=\ttfamily,prebreak={\raisebox{0ex}[0ex][0ex]{\ensuremath{\hookleftarrow}}}]
/* C */
FCSResult fcs_compute_near_potential (FCS handle, fcs_float dist, fcs_float *pot);
\end{lstlisting}
\begin{lstlisting}[language=Fortran,frame=trBL,breaklines,basicstyle=\ttfamily,prebreak={\raisebox{0ex}[0ex][0ex]{\ensuremath{\hookleftarrow}}}]
! Fortran
function fcs_compute_near_potential(handle,dist,pot)
type(c_ptr)                                     ::  handle
real(kind = fcs_real_kind_isoc)                 ::  dist
real(kind = fcs_real_kind_isoc)                 ::  pot
type(c_ptr)                                     ::  fcs_compute_near_potential
\end{lstlisting}
\caption{Function call: fcs\_compute\_near\_potential}
\label{fig:fcs_compute_near_potential}
\ufunctiontoindex{fcs\_compute\_near\_potential}
\end{figure}

\begin{figure}[htb]
\begin{lstlisting}[language=C,frame=trBL,breaklines,basicstyle=\ttfamily,prebreak={\raisebox{0ex}[0ex][0ex]{\ensuremath{\hookleftarrow}}}]
/* C */
FCSResult fcs_compute_near (FCS handle, fcs_float dist, fcs_float *pot, fcs_float *field);
\end{lstlisting}
\begin{lstlisting}[language=Fortran,frame=trBL,breaklines,basicstyle=\ttfamily,prebreak={\raisebox{0ex}[0ex][0ex]{\ensuremath{\hookleftarrow}}}]
! Fortran
function fcs_compute_near(handle,dist,pot,field)
type(c_ptr)                                     ::  handle
real(kind = fcs_real_kind_isoc)                 ::  dist
real(kind = fcs_real_kind_isoc)                 ::  pot
real(kind = fcs_real_kind_isoc), dimension(3)   ::  field
type(c_ptr)                                     ::  fcs_compute_near
\end{lstlisting}
\caption{Function call: fcs\_compute\_near}
\label{fig:fcs_compute_near}
\ufunctiontoindex{fcs\_compute\_near}
\end{figure}

\begin{figure}[htb]
\begin{lstlisting}[language=C,frame=trBL,breaklines,basicstyle=\ttfamily,prebreak={\raisebox{0ex}[0ex][0ex]{\ensuremath{\hookleftarrow}}}]
/* C */
FCSResult fcs_method_has_near (FCS handle, fcs_int *has_near);
\end{lstlisting}
\begin{lstlisting}[language=Fortran,frame=trBL,breaklines,basicstyle=\ttfamily,prebreak={\raisebox{0ex}[0ex][0ex]{\ensuremath{\hookleftarrow}}}]
! Fortran
function fcs_method_has_near(handle,has_near)
type(c_ptr)                                     ::  handle
real(kind = fcs_real_kind_isoc)                 ::  dist
real(kind = fcs_real_kind_isoc), dimension(3)   ::  field
type(c_ptr)                                     ::  fcs_compute_near_field
\end{lstlisting}
\caption{Function call: fcs\_method\_has\_near}
\label{fig:fcs_method_has_near}
\ufunctiontoindex{fcs\_method\_has\_near}
\end{figure}

\todo{fcs\_method\_has\_near fehlt!}

\FloatBarrier
\subsection{Output}

\begin{figure}[htb]
\begin{lstlisting}[language=C,frame=trBL,breaklines,basicstyle=\ttfamily,prebreak={\raisebox{0ex}[0ex][0ex]{\ensuremath{\hookleftarrow}}}]
/* C */
void fcs_printHandle (FCS handle);
\end{lstlisting}
\begin{lstlisting}[language=Fortran,frame=trBL,breaklines,basicstyle=\ttfamily,prebreak={\raisebox{0ex}[0ex][0ex]{\ensuremath{\hookleftarrow}}}]
! Fortran
subroutine fcs_printHandle(handle)
type(c_ptr)                                     ::  handle
\end{lstlisting}
\caption{Function call: fcs\_printHandle}
\label{fig:fcs_printHandle}
\ufunctiontoindex{fcs\_printHandle}
\end{figure}


\FloatBarrier
\subsection{Parser}

\begin{figure}[htb]
\begin{lstlisting}[language=C,frame=trBL,breaklines,basicstyle=\ttfamily,prebreak={\raisebox{0ex}[0ex][0ex]{\ensuremath{\hookleftarrow}}}]
/* C */
FCSResult fcs_parser (FCS handle, const char* parse_string);
\end{lstlisting}
\begin{lstlisting}[language=Fortran,frame=trBL,breaklines,basicstyle=\ttfamily,prebreak={\raisebox{0ex}[0ex][0ex]{\ensuremath{\hookleftarrow}}}]
! Fortran
function fcs_printHandle(handle,parse_string)
type(c_ptr)                                     ::  handle
character(kind = c_char, len = *)               ::  parse_string
type(c_ptr)                                     ::  fcs_parser
\end{lstlisting}
\caption{Function call: fcs\_parser}
\label{fig:fcs_parser}
\ufunctiontoindex{fcs\_parser}
\end{figure}

\FloatBarrier
\subsection{Virial}

\begin{figure}[htb]
\begin{lstlisting}[language=C,frame=trBL,breaklines,basicstyle=\ttfamily,prebreak={\raisebox{0ex}[0ex][0ex]{\ensuremath{\hookleftarrow}}}]
/* C */
FCSResult fcs_require_virial (FCS handle, fcs_int flag);
\end{lstlisting}
\begin{lstlisting}[language=Fortran,frame=trBL,breaklines,basicstyle=\ttfamily,prebreak={\raisebox{0ex}[0ex][0ex]{\ensuremath{\hookleftarrow}}}]
! Fortran
function fcs_require_virial(handle,flag)
type(c_ptr)                                     ::  handle
logical                                         ::  dist
type(c_ptr)                                     ::  fcs_require_virial
\end{lstlisting}
\caption{Function call: fcs\_require\_virial}
\label{fig:fcs_require_virial}
\ufunctiontoindex{fcs\_require\_virial}
\end{figure}

\begin{figure}[htb]
\begin{lstlisting}[language=C,frame=trBL,breaklines,basicstyle=\ttfamily,prebreak={\raisebox{0ex}[0ex][0ex]{\ensuremath{\hookleftarrow}}}]
/* C */
FCSResult fcs_get_virial (FCS handle, fcs_float* virial);
\end{lstlisting}
\begin{lstlisting}[language=Fortran,frame=trBL,breaklines,basicstyle=\ttfamily,prebreak={\raisebox{0ex}[0ex][0ex]{\ensuremath{\hookleftarrow}}}]
! Fortran
function fcs_get_virial(handle,virial)
type(c_ptr)                                     ::  handle
real(kind = fcs_real_kind_isoc), dimension(9)   ::  dist
type(c_ptr)                                     ::  fcs_require_virial
\end{lstlisting}
\caption{Function call: fcs\_get\_virial}
\label{fig:fcs_get_virial}
\ufunctiontoindex{fcs\_get\_virial}
\end{figure}

\FloatBarrier
\section{\direct Solver specific Functions}


\FloatBarrier
\subsection{Getters and Setters}

\FloatBarrier
\section{\ewald Solver specific Functions}


\FloatBarrier
\subsection{Getters and Setters}

\FloatBarrier
\section{\fmm Solver specific Functions}


\FloatBarrier
\subsection{Getters and Setters}

\FloatBarrier
\section{\memd Solver specific Functions}


\FloatBarrier
\subsection{Getters and Setters}

\FloatBarrier
\section{\mmmoned Solver specific Functions}


\FloatBarrier
\subsection{Getters and Setters}

\FloatBarrier
\section{\mmmtwod Solver specific Functions}


\FloatBarrier
\subsection{Getters and Setters}

\FloatBarrier
\section{\pepc Solver specific Functions}


\FloatBarrier
\subsection{Getters and Setters}

\FloatBarrier
\section{\ppthreemg Solver specific Functions}


\FloatBarrier
\subsection{Getters and Setters}

\FloatBarrier
\section{\ptwonfftcap Solver specific Functions}


\FloatBarrier
\subsection{Getters and Setters}

\FloatBarrier
\section{\pthreem Solver specific Functions}

\FloatBarrier
\subsection{Getters and Setters}

\FloatBarrier
\subsection{Near Field Computations}

\begin{figure}[htb]
\begin{lstlisting}[language=C,frame=trBL,breaklines,basicstyle=\ttfamily,prebreak={\raisebox{0ex}[0ex][0ex]{\ensuremath{\hookleftarrow}}}]
/* C */
FCSResult fcs_p3m_compute_near_field (FCS handle, fcs_float dist, fcs_float *field);
\end{lstlisting}
\begin{lstlisting}[language=Fortran,frame=trBL,breaklines,basicstyle=\ttfamily,prebreak={\raisebox{0ex}[0ex][0ex]{\ensuremath{\hookleftarrow}}}]
! Fortran
function fcs_p3m_compute_near_field(handle,dist,field)
type(c_ptr)                                     ::  handle
real(kind = fcs_real_kind_isoc)                 ::  dist
real(kind = fcs_real_kind_isoc), dimension(3)   ::  field
type(c_ptr)                                     ::  fcs_p3m_compute_near_field
\end{lstlisting}
\caption{Function call: fcs\_p3m\_compute\_near\_field}
\label{fig:fcs_p3m_compute_near_field}
\ufunctiontoindex{fcs\_p3m\_compute\_near\_field}
\end{figure}

\begin{figure}[htb]
\begin{lstlisting}[language=C,frame=trBL,breaklines,basicstyle=\ttfamily,prebreak={\raisebox{0ex}[0ex][0ex]{\ensuremath{\hookleftarrow}}}]
/* C */
FCSResult fcs_p3m_compute_near_potential (FCS handle, fcs_float dist, fcs_float *pot);
\end{lstlisting}
\begin{lstlisting}[language=Fortran,frame=trBL,breaklines,basicstyle=\ttfamily,prebreak={\raisebox{0ex}[0ex][0ex]{\ensuremath{\hookleftarrow}}}]
! Fortran
function fcs_p3m_compute_near_potential(handle,dist,pot)
type(c_ptr)                                     ::  handle
real(kind = fcs_real_kind_isoc)                 ::  dist
real(kind = fcs_real_kind_isoc)                 ::  pot
type(c_ptr)                                     ::  fcs_p3m_compute_near_potential
\end{lstlisting}
\caption{Function call: fcs\_p3m\_compute\_near\_potential}
\label{fig:fcs_p3m_compute_near_potential}
\ufunctiontoindex{fcs\_p3m\_compute\_near\_potential}
\end{figure}

\begin{figure}[htb]
\begin{lstlisting}[language=C,frame=trBL,breaklines,basicstyle=\ttfamily,prebreak={\raisebox{0ex}[0ex][0ex]{\ensuremath{\hookleftarrow}}}]
/* C */
FCSResult fcs_p3m_compute_near (FCS handle, fcs_float dist, fcs_float *pot, fcs_float *field);
\end{lstlisting}
\begin{lstlisting}[language=Fortran,frame=trBL,breaklines,basicstyle=\ttfamily,prebreak={\raisebox{0ex}[0ex][0ex]{\ensuremath{\hookleftarrow}}}]
! Fortran
function fcs_p3m_compute_near(handle,dist,pot,field)
type(c_ptr)                                     ::  handle
real(kind = fcs_real_kind_isoc)                 ::  dist
real(kind = fcs_real_kind_isoc)                 ::  pot
real(kind = fcs_real_kind_isoc), dimension(3)   ::  field
type(c_ptr)                                     ::  fcs_p3m_compute_near
\end{lstlisting}
\caption{Function call: fcs\_p3m\_compute\_near}
\label{fig:fcs_p3m_compute_near}
\ufunctiontoindex{fcs\_p3m\_compute\_near}
\end{figure}

\FloatBarrier
\section{VMG Solver specific Functions}

\FloatBarrier
%%% Local Variables: 
%%% mode: latex
%%% TeX-master: ug.tex
%%% End: 
